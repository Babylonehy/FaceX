\chapter{Background and Related Work}
\label{cha:background}

This chapter gives a brief overview of 3D morphable face models and facial expression dataset, the background
knowledge for some facial expression recognition techniques we used.


We firstly introduce widely investigated physiological signals and signal processing
techniques in Section 2.1. Then, the concepts and mathematical derivation of
some time alignment techniques are described in Section 2.2. Following that, the
computation procedure of CNN and LSTM will be introduced. Finally, we provide a
taxonomy for multimodal fusion and introduce their properties.


Section~\ref{sec:motivation} xxxx.\\


Section~\ref{sec:relatedwork} yyyy.\\


\section{Motivation}
\label{sec:motivation}


\section{3D Morphable Face Models}
\label{sec:3dmm}


\section{Facial Expression Dataset}
\label{sec:fed}

\section{Facial Expression Recognition}
\label{sec:fer}


\section{Summary}
Summary what you discussed in this chapter, and mention the story in next
chapter. Readers should roughly understand what your thesis takes about by only reading
words at the beginning and the end (Summary) of each chapter.



