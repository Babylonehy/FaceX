\chapter{Background and Related Work}
\label{cha:background}

This chapter gives a brief overview of 3D morphable face models and facial expression dataset, the background knowledge for some facial expression recognition techniques we used.

We firstly introduce widely investigated Facial Expression Recognition in Section \ref{sec:fer}. Then, the public availability facial expression datasets are described in Section \ref{sec:fed}. Finally, we provide a
review of building and applying 3D morphable face models \ref{sec:3dmm}.

\section{Facial Expression Recognition}
\label{sec:fer}

\subsection{Traditional methods}
\subsubsection{Feature Extraction}

\subsubsection{Classification}

\subsection{Learn-based Method}

\subsubsection{CNN-based Approaches}
\subsubsection{Long Short-Term Memory(LSTM)}
\subsubsection{Genrative Adversarial Network (GAN)}

\section{Facial Expression Dataset}
\label{sec:fed}

\subsection{Vedio datasets}

\subsection{Image datasets}

\section{3D Morphable Face Models}
\label{sec:3dmm}
\subsection{Shape model}
\subsection{Expression model}
\subsection{Appearance model}


\section{Summary}
% Summary what you discussed in this chapter, and mention the story in next
% chapter. Readers should roughly understand what your thesis takes about by only reading
% words at the beginning and the end (Summary) of each chapter.



