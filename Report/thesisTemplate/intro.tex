\chapter{Introduction}
\label{cha:intro}


\section{Thesis Statement}
\label{sec:thesisstatement}
FaceX is a lightweight, flexible and scalable engine that can be used to generate a variety of high-quality images with diversity of facial expressions and shapes. 


\section{Introduction}
\label{sec:problemstatement}
% Put your introduction here. You could use \textbackslash fix\{ABCDEFG.\} to
% leave your comments, see the box at the left side. \fix{You have to rewrite your
% thesis!!!}

In the last few decades, facial expression analysis (FEA) is a challenging task of computer vision and has attracted the interest of more and more researchers. Facial expression has been proven to play important role in understanding human emotion \citep{Mehrabian_Russell_1974}. Because facial expression is the response of a person's mental state to external stimuli \citep{Cabanac_2002}. And \citeauthor{ekmanArgumentBasicEmotions1992} classifies human emotions into seven basic categories: Happy, Sad, Surprise, Anger, Disgust, Fear and Neutral.

With the development of artificial intelligence, especially machine learning (ML), people benefit from artificial intelligence agents adjusting their response according to their emotional state \citep{adolphsInvestigatingEmotionsFunctional2018}. In this regard, 
there are braod facial expression applications in different domains like Human-Centred Computing(Hcc) \citep{cowieEmotionRecognitionHumancomputer2001}, augmented reality (AR) \citep{chenAugmentedRealitybasedSelffacial2015}, virtual reality (VR) \citep{bekeleUnderstandingHowAdolescents2013}, automatic driving \citep{jabonFacialExpressionAnalysis2011}, and gaming \citep{lankesFacialExpressionsGame2008}. 

Various types of data can feed the FEA systems. In computer vision, facial images are the mainstream input data type. In addition, electromyography (EMG), electrocardiographic(ECG) and other related physical or chemical signal can be used as input data or auxiliary data as well \citep{jerrittaPhysiologicalSignalsBased2011}. This thesis focus on using facial images taken by sensor to detect expression and feeling of people. Because facial images contain sufficient non-verbal information for FEA \citep{huangFacialExpressionRecognition2019}. 

We briefly review the development of FEA in computer vision, most of FEA techniques can be defined as either traditional methods or learning-based methods \citep{huangFacialExpressionRecognition2019}.

\textbf{Traditional methods} involves various hand-craft feature. Those approaches need to design appropriate 

\section{Thesis Outline}
\label{sec:outline}
How many chapters you have? You may have Chapter~\ref{cha:background},
Chapter~\ref{cha:design}, Chapter~\ref{cha:methodology},
Chapter~\ref{cha:result}, and Chapter~\ref{cha:conc}.
